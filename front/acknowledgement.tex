% !TeX root = ../main.tex

\begin{acknowledgement}
	不知不覺,已經步入生態學的世界四年了。自從在大三的時候修了五木老師的樹木學,開始燃起我對於自然世界的興趣,也促使我去參加了2019年的福山森林動態樣區的每木調查。在福山跟家豪聊過之後,我才發現生態的研究是有用到蠻多數學的,讓我覺得自己的數學背景並不會讓我在這個領域太過突兀。調查結束之後,在台大找到了大衛的研究室,這才真正的開始我在生態領域中的旅程。
	
	真的非常感謝我的指導教授David,每次都花很多心力和腦力在跟我討論我的研究方向和分析結果。這兩年半來,他帶我認識生態研究中的許多不同面相,教我如何嚴謹的做研究、思考問題和挖掘每個問題背後的價值。在分析資料的部分,我借用研究室的電腦,好讓我跑分析的時候能有足夠的記憶體來處理龐大的資料量。在撰寫論文的時候,David也很用心的給我建議,並跟我討論要怎麼寫才能讓句子變得精準但又好懂。David同時也是一位很棒的人生導師,他在我面對人生的無常的時候安慰我,也告訴我要去好好接納和感受自己的情緒和想法,這些建議始終影響和造就著現在我。同時,我也感謝柯柏如老師和我討論撰寫論文的大方向。感謝張楊家豪老師讓我使用福山四年的資料,來演示我的方法。感謝Corentin老師願意接受台美遠距口試,給了我很多發表上的建議。另外也感謝魏碩、宇霈、晟洋、書逸、訓宏、尹舜和其他研究室的夥伴時常在能一起討論生態問題或是吃飯聊聊天,讓研究所的生活不這麼苦悶。感謝柏佑在口試前後幫助我很多在行政上的事。感謝亭羽和其他朋友、家人,在寫論文的時候陪伴我或是給予支持,讓我能順利的度過這個焦慮的時期。
	
	看越多文獻,就越發現在生態領域中有很多不同面向的研究,也有許多未解的問題。這個世界越看越覺得廣大和有趣,期許自己能成為一個站在最前線、探索未知的世界的冒險者。

\end{acknowledgement}