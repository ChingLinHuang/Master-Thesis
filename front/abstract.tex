% !TeX root = ../main.tex

\begin{abstract}

關聯群落(metacommunity)由多個群落所組成,並且被四種生態機制所影響:競爭(competition)、環境過濾(environmental filtering)、遷徙(dispersal)和隨機性(stochasticity)。許多方法被用來量化群落組成背後的機制,但僅有少數研究將這些方法整合起來。Guzman等人(2022)整合多種分析方法來量化群落組成背後的機制並使用模擬資料評估其效力。雖然他們的研究整合了多種分析方法,並總結只有結合多種分析方法並應用在完整的關聯群落資料上,才有較高的準確率去量化機制,但他們沒有提出其方法能否應用在實際資料上。本研究討論了Guzman等人(2022)的方法能否應用在實際的生態研究中。我們利用關聯群落模型來產生模擬資料,其模型考慮了非生物和生物間的交互作用以及遷徙等機制。透過調整模型的參數,我們可以改變機制的強度。本研究考慮了三種分析方法:beta-diversity variation partitioning、Stegen方法和dispersal niche continuum index,並利用隨機森林(random forest)將這些分析方法產生的統計量與模擬模型的參數連結起來。根據模擬資料,本研究顯示,越多的分析方法和完整的時間尺度資料被結合起來,估計參數的準確率就會變高。本研究也發現,當資料不完整時,我們對於機制的估計會變得不準確,與Guzman等人(2022)指出的相同。本研究另外展示了Guzman等人(2022)的方法在福山森林動態樣區的應用,發現在福山的物種之間擁有競爭和遷徙之間的權衡(competition-colonization trade-off),並探討在使用其方法時需要考慮的地方。


\end{abstract}

\begin{abstract*}
\noindent
A metacommunity is a set of interconnected communities that incorporate multiple co-occurring species with different abundances. The species composition of a metacommunity is shaped by four main ecological processes: competition, environmental filtering, dispersal and stochasticity. Ecologists have proposed several analytical methods, such as beta-diversity variation partitioning, to quantify the relative importance of ecological processes in shaping metacommunity composition based on information derived from metacommunity composition structure. However, these analytical methods were rarely synthesized. Guzman et al. (2022) integrated multiple analytical methods to estimate the relative importance of ecological processes and evaluated their framework based on the simulated data generated by a process-based simulation model. They concluded that integrating multiple analytical methods and high completeness of the metacommunity across space and time will improve the correctness of the estimation of process-based model parameters. However, the authors did not discuss whether their framework could be applied to observational data. In our study, we reconstructed Guzman et al.'s framework and discussed its application value to the empirical community data. We used the same process-based model as Guzman et al. to generate simulated metacommunity data, incorporating abiotic and biotic interactions, dispersal, and stochasticity. We manipulated the parameters of these four processes to generate metacommunity scenarios with different strengths of underlying processes. We used three analytical methods to calculate summary statistics of the simulated data: 1) beta-diversity variation partitioning, 2) Stegen's framework, and 3) dispersal-niche continuum index (DNCI). We then used the random forest algorithm to estimate the parameters of the process-based model and disentangle different metacommunity scenarios based on the summary statistics. Based on the simulated data, we showed that if the random forest model incorporated the summary statistics derived from more analytical methods and more snapshots of the species composition, it will have better accuracy for estimating the model parameters. We also showed that the incompleteness of the species composition data will decrease the accuracy in estimating the model parameters. On the contrary, the accuracy was not influenced by the choice of the snapshots in the simulation. We also illustrated the application of the trained random forest by analyzing repeated census of woody plant species in the Fushan Forest Dynamics Plot and showed that the community of this forest is based mainly on competition-colonization trade-off. We also discussed what needs to be considered when Guzman et al.'s framework is applied. We conclude that Guzman et al.'s framework based on integrated analytical methods could be successfully applied to observational studies to disentangle the ecological processes shaping observed metacommunity.

\end{abstract*}