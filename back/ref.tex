\begin{thebibliography}{70}
	\expandafter\ifx\csname natexlab\endcsname\relax\def\natexlab#1{#1}\fi
	
	\bibitem[{Adler \emph{et~al.}(2010)Adler, Ellner \&
		Levine}]{adler2010coexistence}
	Adler, P.B., Ellner, S.P. \& Levine, J.M. (2010).
	\newblock Coexistence of perennial plants: an embarrassment of niches.
	\newblock \emph{Ecology Letters}, 13, 1019--1029.
	
	\bibitem[{Adler \emph{et~al.}(2007)Adler, HilleRisLambers \&
		Levine}]{adler2007niche}
	Adler, P.B., HilleRisLambers, J. \& Levine, J.M. (2007).
	\newblock A niche for neutrality.
	\newblock \emph{Ecology Letters}, 10, 95--104.
	
	\bibitem[{Bezanson \emph{et~al.}(2017)Bezanson, Edelman, Karpinski \&
		Shah}]{bezanson2017julia}
	Bezanson, J., Edelman, A., Karpinski, S. \& Shah, V.B. (2017).
	\newblock Julia: A fresh approach to numerical computing.
	\newblock \emph{SIAM Review}, 59, 65--98.
	
	\bibitem[{Borcard \& Legendre(2002)}]{borcard2002all}
	Borcard, D. \& Legendre, P. (2002).
	\newblock All-scale spatial analysis of ecological data by means of principal
	coordinates of neighbour matrices.
	\newblock \emph{Ecological Modelling}, 153, 51--68.
	
	\bibitem[{Borcard \emph{et~al.}(1992)Borcard, Legendre \&
		Drapeau}]{borcard1992partialling}
	Borcard, D., Legendre, P. \& Drapeau, P. (1992).
	\newblock Partialling out the spatial component of ecological variation.
	\newblock \emph{Ecology}, 73, 1045--1055.

	\bibitem[{Borics \emph{et~al.}(2020)Borics, B-B{\'e}res, B{\'a}csi, Luk{\'a}cs,
  	T-Krasznai, Botta-Duk{\'a}t \& V{\'a}rb{\'\i}r{\'o}}]{borics2020trait}
	Borics, G., B-B{\'e}res, V., B{\'a}csi, I., Luk{\'a}cs, B.A., T-Krasznai, E.,
  	Botta-Duk{\'a}t, Z. \& V{\'a}rb{\'\i}r{\'o}, G. (2020).
	\newblock Trait convergence and trait divergence in lake phytoplankton reflect
  	community assembly rules.
	\newblock \emph{Scientific Reports}, 10, 19599.
	
	\bibitem[{ter Braak(1986)}]{ter1986canonical}
	ter Braak, C.J. (1986).
	\newblock Canonical correspondence analysis: a new eigenvector technique for
	multivariate direct gradient analysis.
	\newblock \emph{Ecology}, 67, 1167--1179.
	
	\bibitem[{Brown \emph{et~al.}(2017)Brown, Sokol, Skelton \&
		Tornwall}]{brown2017making}
	Brown, B.L., Sokol, E.R., Skelton, J. \& Tornwall, B. (2017).
	\newblock Making sense of metacommunities: dispelling the mythology of a
	metacommunity typology.
	\newblock \emph{Oecologia}, 183, 643--652.
	
	\bibitem[{Chang \emph{et~al.}(2013)Chang, Zelený, Li, Chiu \&
		Hsieh}]{chang2013better}
	Chang, L.-W., Zelený, D., Li, C.-F., Chiu, S.-T. \& Hsieh, C.-F. (2013).
	\newblock Better environmental data may reverse conclusions about niche-and
	dispersal-based processes in community assembly.
	\newblock \emph{Ecology}, 94, 2145--2151.
	
	\bibitem[{Chase \emph{et~al.}(2020)Chase, Jeliazkov, Ladouceur \&
		Viana}]{chase2020biodiversity}
	Chase, J.M., Jeliazkov, A., Ladouceur, E. \& Viana, D.S. (2020).
	\newblock Biodiversity conservation through the lens of metacommunity ecology.
	\newblock \emph{Annals of the New York Academy of Sciences}, 1469, 86--104.
	
	\bibitem[{Chase \& Myers(2011)}]{chase2011disentangling}
	Chase, J.M. \& Myers, J.A. (2011).
	\newblock Disentangling the importance of ecological niches from stochastic
	processes across scales.
	\newblock \emph{Philosophical Transactions of the Royal Society B: Biological
		Sciences}, 366, 2351--2363.
	
	\bibitem[{Chave \emph{et~al.}(2009)Chave, Coomes, Jansen, Lewis, Swenson \&
		Zanne}]{chave2009towards}
	Chave, J., Coomes, D., Jansen, S., Lewis, S.L., Swenson, N.G. \& Zanne, A.E.
	(2009).
	\newblock Towards a worldwide wood economics spectrum.
	\newblock \emph{Ecology Letters}, 12, 351--366.
	
	\bibitem[{Chave \emph{et~al.}(2002)Chave, Muller-Landau \&
		Levin}]{chave2002comparing}
	Chave, J., Muller-Landau, H.C. \& Levin, S.A. (2002).
	\newblock Comparing classical community models: theoretical consequences for
	patterns of diversity.
	\newblock \emph{The American Naturalist}, 159, 1--23.
	
	\bibitem[{Chesson(2000)}]{chesson2000mechanisms}
	Chesson, P. (2000).
	\newblock Mechanisms of maintenance of species diversity.
	\newblock \emph{Annual Review of Ecology and Systematics}, pp. 343--366.
	
	\bibitem[{Clark \emph{et~al.}(2001)Clark, Carpenter, Barber, Collins, Dobson,
		Foley, Lodge, Pascual, Pielke~Jr, Pizer \emph{et~al.}}]{clark2001ecological}
	Clark, J.S., Carpenter, S.R., Barber, M., Collins, S., Dobson, A., Foley, J.A.,
	Lodge, D.M., Pascual, M., Pielke~Jr, R., Pizer, W. \emph{et~al.} (2001).
	\newblock Ecological forecasts: an emerging imperative.
	\newblock \emph{Science}, 293, 657--660.
	
	\bibitem[{Clarke(1993)}]{clarke1993non}
	Clarke, K.R. (1993).
	\newblock Non-parametric multivariate analyses of changes in community
	structure.
	\newblock \emph{Australian Journal of Ecology}, 18, 117--143.
	
	\bibitem[{Condit(1998)}]{condit1998tropical}
	Condit, R. (1998).
	\newblock \emph{Tropical forest census plots: methods and results from Barro
		Colorado Island, Panama and a comparison with other plots}.
	\newblock Springer Science \& Business Media.
	
	\bibitem[{Connolly \emph{et~al.}(2017)Connolly, Keith, Colwell \&
		Rahbek}]{connolly2017process}
	Connolly, S.R., Keith, S.A., Colwell, R.K. \& Rahbek, C. (2017).
	\newblock Process, mechanism, and modeling in macroecology.
	\newblock \emph{Trends in Ecology \& Evolution}, 32, 835--844.
	
	\bibitem[{Connor \& Simberloff(1979)}]{connor1979assembly}
	Connor, E.F. \& Simberloff, D. (1979).
	\newblock The assembly of species communities: chance or competition?
	\newblock \emph{Ecology}, 60, 1132--1140.
	
	\bibitem[{Cottenie(2005)}]{cottenie2005integrating}
	Cottenie, K. (2005).
	\newblock Integrating environmental and spatial processes in ecological
	community dynamics.
	\newblock \emph{Ecology Letters}, 8, 1175--1182.
	
	\bibitem[{Csill{\'e}ry \emph{et~al.}(2010)Csill{\'e}ry, Blum, Gaggiotti \&
		Fran{\c{c}}ois}]{csillery2010approximate}
	Csill{\'e}ry, K., Blum, M.G., Gaggiotti, O.E. \& Fran{\c{c}}ois, O. (2010).
	\newblock Approximate bayesian computation (abc) in practice.
	\newblock \emph{Trends in Ecology \& Evolution}, 25, 410--418.
	
	\bibitem[{Diamond(1975)}]{diamond1975island}
	Diamond, J.M. (1975).
	\newblock The island dilemma: lessons of modern biogeographic studies for the
	design of natural reserves.
	\newblock \emph{Biological Conservation}, 7, 129--146.
	
	\bibitem[{Evans(2012)}]{evans2012modelling}
	Evans, M.R. (2012).
	\newblock Modelling ecological systems in a changing world.
	\newblock \emph{Philosophical Transactions of the Royal Society B: Biological
		Sciences}, 367, 181--190.
	
	\bibitem[{Ford \& Roberts(2020)}]{ford2020functional}
	Ford, B.M. \& Roberts, J.D. (2020).
	\newblock Functional traits reveal the presence and nature of multiple
	processes in the assembly of marine fish communities.
	\newblock \emph{Oecologia}, 192, 143--154.
	
	\bibitem[{Fukami(2015)}]{fukami2015historical}
	Fukami, T. (2015).
	\newblock Historical contingency in community assembly: integrating niches,
	species pools, and priority effects.
	\newblock \emph{Annual Review of Ecology, Evolution, and Systematics}, 46,
	1--23.
	
	\bibitem[{Gibert \& Escarguel(2019)}]{gibert2019per}
	Gibert, C. \& Escarguel, G. (2019).
	\newblock PER-SIMPER—a new tool for inferring community assembly processes
	from taxon occurrences.
	\newblock \emph{Global Ecology and Biogeography}, 28, 374--385.
	
	\bibitem[{Gotelli \& McGill(2006)}]{gotelli2006null}
	Gotelli, N.J. \& McGill, B.J. (2006).
	\newblock Null versus neutral models: what's the difference?
	\newblock \emph{Ecography}, 29, 793--800.
	
	\bibitem[{Gotelli \& Ulrich(2012)}]{gotelli2012statistical}
	Gotelli, N.J. \& Ulrich, W. (2012).
	\newblock Statistical challenges in null model analysis.
	\newblock \emph{Oikos}, 121, 171--180.
	
	\bibitem[{Gravel \emph{et~al.}(2006)Gravel, Canham, Beaudet \&
		Messier}]{gravel2006reconciling}
	Gravel, D., Canham, C.D., Beaudet, M. \& Messier, C. (2006).
	\newblock Reconciling niche and neutrality: the continuum hypothesis.
	\newblock \emph{Ecology Letters}, 9, 399--409.
	
	\bibitem[{Guzman \emph{et~al.}(2022)Guzman, Thompson, Viana, Vanschoenwinkel,
		Horv{\'a}th, Ptacnik, Jeliazkov, Gasc{\'o}n, Lemmens, Anton-Pardo
		\emph{et~al.}}]{guzman2022accounting}
	Guzman, L.M., Thompson, P.L., Viana, D.S., Vanschoenwinkel, B., Horv{\'a}th,
	Z., Ptacnik, R., Jeliazkov, A., Gasc{\'o}n, S., Lemmens, P., Anton-Pardo, M.
	\emph{et~al.} (2022).
	\newblock Accounting for temporal change in multiple biodiversity patterns
	improves the inference of metacommunity processes.
	\newblock \emph{Ecology}, 103.
	
	\bibitem[{Han \emph{et~al.}(2016)Han, Guo \& Yu}]{han2016variable}
	Han, H., Guo, X. \& Yu, H. (2016).
	\newblock Variable selection using mean decrease accuracy and mean decrease
	gini based on random forest.
	\newblock In: \emph{2016 7th IEEE International Conference on Software
		Engineering and Service Science (ICSESS)}. IEEE, pp. 219--224.
	
	\bibitem[{Hartig \emph{et~al.}(2011)Hartig, Calabrese, Reineking, Wiegand \&
		Huth}]{hartig2011statistical}
	Hartig, F., Calabrese, J.M., Reineking, B., Wiegand, T. \& Huth, A. (2011).
	\newblock Statistical inference for stochastic simulation models--theory and
	application.
	\newblock \emph{Ecology Letters}, 14, 816--827.
	
	\bibitem[{Hodgson \& Halpern(2019)}]{hodgson2019investigating}
	Hodgson, E.E. \& Halpern, B.S. (2019).
	\newblock Investigating cumulative effects across ecological scales.
	\newblock \emph{Conservation Biology}, 33, 22--32.
	
	\bibitem[{Hornung(2020)}]{hornung2020ordinal}
	Hornung, R. (2020).
	\newblock Ordinal forests.
	\newblock \emph{Journal of Classification}, 37, 4--17.
	
	\bibitem[{Hubbell(2011)}]{hubbell2011unified}
	Hubbell, S.P. (2011).
	\newblock \emph{The Unified Neutral Theory of Biodiversity and Biogeography}.
	\newblock Princeton University Press.
	
	\bibitem[{Janitza \emph{et~al.}(2016)Janitza, Tutz \&
		Boulesteix}]{janitza2016random}
	Janitza, S., Tutz, G. \& Boulesteix, A.L. (2016).
	\newblock Random forest for ordinal responses: prediction and variable
	selection.
	\newblock \emph{Computational Statistics \& Data Analysis}, 96, 57--73.
	
	\bibitem[{Ke \emph{et~al.}(2015)Ke, Miki \& Ding}]{ke2015soil}
	Ke, P.-J., Miki, T. \& Ding, T.-S. (2015).
	\newblock The soil microbial community predicts the importance of plant traits
	in plant--soil feedback.
	\newblock \emph{New Phytologist}, 206, 329--341.
	
	\bibitem[{Kraft \emph{et~al.}(2015)Kraft, Adler, Godoy, James, Fuller \&
		Levine}]{kraft2015community}
	Kraft, N.J., Adler, P.B., Godoy, O., James, E.C., Fuller, S. \& Levine, J.M.
	(2015).
	\newblock Community assembly, coexistence and the environmental filtering
	metaphor.
	\newblock \emph{Functional Ecology}, 29, 592--599.
	
	\bibitem[{Kraft \emph{et~al.}(2011)Kraft, Comita, Chase, Sanders, Swenson,
		Crist, Stegen, Vellend, Boyle, Anderson
		\emph{et~al.}}]{kraft2011disentangling}
	Kraft, N.J., Comita, L.S., Chase, J.M., Sanders, N.J., Swenson, N.G., Crist,
	T.O., Stegen, J.C., Vellend, M., Boyle, B., Anderson, M.J. \emph{et~al.}
	(2011).
	\newblock Disentangling the drivers of $\beta$ diversity along latitudinal and
	elevational gradients.
	\newblock \emph{Science}, 333, 1755--1758.
	
	\bibitem[{Legendre \& Legendre(2012)}]{legendre2012numerical}
	Legendre, P. \& Legendre, L. (2012).
	\newblock \emph{Numerical ecology}.
	\newblock Elsevier.
	
	\bibitem[{Leibold \& Chase(2017)}]{leibold2017metacommunity}
	Leibold, M.A. \& Chase, J.M. (2017).
	\newblock \emph{Metacommunity Ecology}.
	\newblock Princeton University Press.
	
	\bibitem[{Leibold \emph{et~al.}(2004)Leibold, Holyoak, Mouquet, Amarasekare,
		Chase, Hoopes, Holt, Shurin, Law, Tilman
		\emph{et~al.}}]{leibold2004metacommunity}
	Leibold, M.A., Holyoak, M., Mouquet, N., Amarasekare, P., Chase, J.M., Hoopes,
	M.F., Holt, R.D., Shurin, J.B., Law, R., Tilman, D. \emph{et~al.} (2004).
	\newblock The metacommunity concept: a framework for multi-scale community
	ecology.
	\newblock \emph{Ecology Letters}, 7, 601--613.
	
	\bibitem[{Levins(1966)}]{levins1966strategy}
	Levins, R. (1966).
	\newblock The strategy of model building in population biology.
	\newblock \emph{American Scientist}, 54, 421--431.
	
	\bibitem[{MacArthur \& Levins(1967)}]{macarthur1967limiting}
	MacArthur, R. \& Levins, R. (1967).
	\newblock The limiting similarity, convergence, and divergence of coexisting
	species.
	\newblock \emph{The American Naturalist}, 101, 377--385.
	
	\bibitem[{MacArthur(1958)}]{macarthur1958population}
	MacArthur, R.H. (1958).
	\newblock Population ecology of some warblers of northeastern coniferous
	forests.
	\newblock \emph{Ecology}, 39, 599--619.
	
	\bibitem[{MacArthur \& Wilson(1967)}]{macarthur1967theory}
	MacArthur, R.H. \& Wilson, E.O. (1967).
	\newblock \emph{The Theory of Island Biogeography}.
	\newblock Princeton University Press.
	
	\bibitem[{Mayfield \& Levine(2010)}]{mayfield2010opposing}
	Mayfield, M.M. \& Levine, J.M. (2010).
	\newblock Opposing effects of competitive exclusion on the phylogenetic
	structure of communities.
	\newblock \emph{Ecology Letters}, 13, 1085--1093.
	
	\bibitem[{McGill(2010)}]{mcgill2010towards}
	McGill, B.J. (2010).
	\newblock Towards a unification of unified theories of biodiversity.
	\newblock \emph{Ecology Letters}, 13, 627--642.
	
	\bibitem[{McGill \emph{et~al.}(2006)McGill, Maurer \&
		Weiser}]{mcgill2006empirical}
	McGill, B.J., Maurer, B.A. \& Weiser, M.D. (2006).
	\newblock Empirical evaluation of neutral theory.
	\newblock \emph{Ecology}, 87, 1411--1423.
	
	\bibitem[{Molina \& Stone(2020)}]{molina2020difficulties}
	Molina, C. \& Stone, L. (2020).
	\newblock Difficulties in benchmarking ecological null models: an assessment of
	current methods.
	\newblock \emph{Ecology}, 101, e02945.
	
	\bibitem[{Ning \emph{et~al.}(2019)Ning, Deng, Tiedje \& Zhou}]{ning2019general}
	Ning, D., Deng, Y., Tiedje, J.M. \& Zhou, J. (2019).
	\newblock A general framework for quantitatively assessing ecological
	stochasticity.
	\newblock \emph{Proceedings of the National Academy of Sciences}, 116,
	16892--16898.
	
	\bibitem[{Ovaskainen \emph{et~al.}(2017)Ovaskainen, Tikhonov, Norberg,
		Guillaume~Blanchet, Duan, Dunson, Roslin \& Abrego}]{ovaskainen2017make}
	Ovaskainen, O., Tikhonov, G., Norberg, A., Guillaume~Blanchet, F., Duan, L.,
	Dunson, D., Roslin, T. \& Abrego, N. (2017).
	\newblock How to make more out of community data? a conceptual framework and
	its implementation as models and software.
	\newblock \emph{Ecology Letters}, 20, 561--576.
	
	\bibitem[{Peres-Neto \emph{et~al.}(2006)Peres-Neto, Legendre, Dray \&
		Borcard}]{peres2006variation}
	Peres-Neto, P.R., Legendre, P., Dray, S. \& Borcard, D. (2006).
	\newblock Variation partitioning of species data matrices: estimation and
	comparison of fractions.
	\newblock \emph{Ecology}, 87, 2614--2625.
	
	\bibitem[{Perez-Harguindeguy \emph{et~al.}(2013)Perez-Harguindeguy, Diaz,
		Garnier, Lavorel, Poorter, Jaureguiberry, Bret-Harte, Cornwell, Craine,
		Gurvich, Urcelay, Veneklaas, Reich, Poorter, Wright, Ray, Enrico, Pausas,
		de~Vos, Buchmann, Funes, Quetier, Hodgson, Thompson, Morgan, ter Steege,
		van~der Heijden, Sack, Blonder, Poschlod, Vaieretti, Conti, Staver, Aquino \&
		Cornelissen}]{hrguindeguy2013new}
	Perez-Harguindeguy, N., Diaz, S., Garnier, E., Lavorel, S., Poorter, H.,
	Jaureguiberry, P., Bret-Harte, M.S., Cornwell, W.K., Craine, J.M., Gurvich,
	D.E., Urcelay, C., Veneklaas, E.J., Reich, P.B., Poorter, L., Wright, I.J.,
	Ray, P., Enrico, L., Pausas, J.G., de~Vos, A.C., Buchmann, N., Funes, G.,
	Quetier, F., Hodgson, J.G., Thompson, K., Morgan, H.D., ter Steege, H.,
	van~der Heijden, M.G.A., Sack, L., Blonder, B., Poschlod, P., Vaieretti,
	M.V., Conti, G., Staver, A.C., Aquino, S. \& Cornelissen, J.H.C. (2013).
	\newblock New handbook for standardised measurement of plant functional traits
	worldwide.
	\newblock \emph{Australian Journal of Botany}, 61, 167--234.
	
	\bibitem[{van~der Plas \emph{et~al.}(2015)van~der Plas, Janzen, Ordonez,
		Fokkema, Reinders, Etienne \& Olff}]{van2015new}
	van~der Plas, F., Janzen, T., Ordonez, A., Fokkema, W., Reinders, J., Etienne,
	R.S. \& Olff, H. (2015).
	\newblock A new modeling approach estimates the relative importance of
	different community assembly processes.
	\newblock \emph{Ecology}, 96, 1502--1515.
	
	\bibitem[{{R Core Team}(2022)}]{R}
	{R Core Team} (2022).
	\newblock \emph{R: A Language and Environment for Statistical Computing}.
	\newblock R Foundation for Statistical Computing, Vienna, Austria.
	
	\bibitem[{Ron \emph{et~al.}(2018)Ron, Fragman-Sapir \&
		Kadmon}]{ron2018dispersal}
	Ron, R., Fragman-Sapir, O. \& Kadmon, R. (2018).
	\newblock Dispersal increases ecological selection by increasing effective
	community size.
	\newblock \emph{Proceedings of the National Academy of Sciences}, 115,
	11280--11285.
	
	\bibitem[{Ruffley \emph{et~al.}(2019)Ruffley, Peterson, Week, Tank \&
		Harmon}]{ruffley2019identifying}
	Ruffley, M., Peterson, K., Week, B., Tank, D.C. \& Harmon, L.J. (2019).
	\newblock Identifying models of trait-mediated community assembly using random
	forests and approximate bayesian computation.
	\newblock \emph{Ecology and Evolution}, 9, 13218--13230.
	
	\bibitem[{Schlather \emph{et~al.}(2015)Schlather, Malinowski, Menck, Oesting \&
		Strokorb}]{schlather2015analysis}
	Schlather, M., Malinowski, A., Menck, P.J., Oesting, M. \& Strokorb, K. (2015).
	\newblock Analysis, simulation and prediction of multivariate random fields
	with package random fields.
	\newblock \emph{Journal of Statistical Software}, 63, 1--25.
	
	\bibitem[{S{\^\i}rbu \emph{et~al.}(2021)S{\^\i}rbu, Benedek \&
		S{\^\i}rbu}]{sirbu2021variation}
	S{\^\i}rbu, I., Benedek, A.M. \& S{\^\i}rbu, M. (2021).
	\newblock Variation partitioning in double-constrained multivariate analyses:
	linking communities, environment, space, functional traits, and ecological
	niches.
	\newblock \emph{Oecologia}, 197, 43--59.
	
	\bibitem[{Smith \& Lundholm(2010)}]{smith2010variation}
	Smith, T.W. \& Lundholm, J.T. (2010).
	\newblock Variation partitioning as a tool to distinguish between niche and
	neutral processes.
	\newblock \emph{Ecography}, 33, 648--655.
	
	\bibitem[{Stegen \emph{et~al.}(2013)Stegen, Lin, Fredrickson, Chen, Kennedy,
		Murray, Rockhold \& Konopka}]{stegen2013quantifying}
	Stegen, J.C., Lin, X., Fredrickson, J.K., Chen, X., Kennedy, D.W., Murray,
	C.J., Rockhold, M.L. \& Konopka, A. (2013).
	\newblock Quantifying community assembly processes and identifying features
	that impose them.
	\newblock \emph{The ISME Journal}, 7, 2069--2079.
	
	\bibitem[{Su \emph{et~al.}(2007)Su, Chang-Yang, Lu, Tsui, Lin, Lin, Chiou,
		Kuan, Chen \& Hsieh}]{su2007fushan}
	Su, S.-H., Chang-Yang, C.-H., Lu, C.-L., Tsui, C.-C., Lin, T.-T., Lin, C.-L., Chiou,
	W.-L., Kuan, L.-H., Chen, Z.-S. \& Hsieh, C.-F. (2007).
	\newblock \emph{Fushan Subtropical Forest Dynamics Plot: Tree Species
		Characteristics and Distribution Patterns}.
	\newblock Taiwan Forestry Research Institute.
	
	\bibitem[{Thompson \emph{et~al.}(2020)Thompson, Guzman, De~Meester,
		Horv{\'a}th, Ptacnik, Vanschoenwinkel, Viana \& Chase}]{thompson2020process}
	Thompson, P.L., Guzman, L.M., De~Meester, L., Horv{\'a}th, Z., Ptacnik, R.,
	Vanschoenwinkel, B., Viana, D.S. \& Chase, J.M. (2020).
	\newblock A process-based metacommunity framework linking local and regional
	scale community ecology.
	\newblock \emph{Ecology Letters}, 23, 1314--1329.
	
	\bibitem[{Tilman(1997)}]{tilman1997community}
	Tilman, D. (1997).
	\newblock Community invasibility, recruitment limitation, and grassland
	biodiversity.
	\newblock \emph{Ecology}, 78, 81--92.
	
	\bibitem[{Tucker \emph{et~al.}(2016)Tucker, Shoemaker, Davies, Nemergut \&
		Melbourne}]{tucker2016differentiating}
	Tucker, C.M., Shoemaker, L.G., Davies, K.F., Nemergut, D.R. \& Melbourne, B.A.
	(2016).
	\newblock Differentiating between niche and neutral assembly in metacommunities
	using null models of $\beta$-diversity.
	\newblock \emph{Oikos}, 125, 778--789.
	
	\bibitem[{Ulrich \& Gotelli(2010)}]{ulrich2010null}
	Ulrich, W. \& Gotelli, N.J. (2010).
	\newblock Null model analysis of species associations using abundance data.
	\newblock \emph{Ecology}, 91, 3384--3397.
	
	\bibitem[{Vellend \emph{et~al.}(2014)Vellend, Srivastava, Anderson, Brown,
		Jankowski, Kleynhans, Kraft, Letaw, Macdonald, Maclean
		\emph{et~al.}}]{vellend2014assessing}
	Vellend, M., Srivastava, D.S., Anderson, K.M., Brown, C.D., Jankowski, J.E.,
	Kleynhans, E.J., Kraft, N.J., Letaw, A.D., Macdonald, A.A.M., Maclean, J.E.
	\emph{et~al.} (2014).
	\newblock Assessing the relative importance of neutral stochasticity in
	ecological communities.
	\newblock \emph{Oikos}, 123, 1420--1430.
	
	\bibitem[{Vilmi \emph{et~al.}(2021)Vilmi, Gibert, Escarguel, Happonen, Heino,
		Jamoneau, Passy, Picazo, Soininen, Tison-Rosebery
		\emph{et~al.}}]{vilmi2021dispersal}
	Vilmi, A., Gibert, C., Escarguel, G., Happonen, K., Heino, J., Jamoneau, A.,
	Passy, S.I., Picazo, F., Soininen, J., Tison-Rosebery, J. \emph{et~al.}
	(2021).
	\newblock Dispersal--niche continuum index: a new quantitative metric for
	assessing the relative importance of dispersal versus niche processes in
	community assembly.
	\newblock \emph{Ecography}, 44, 370--379.
	
	\bibitem[{Wilson(1992)}]{wilson1992complex}
	Wilson, D.S. (1992).
	\newblock Complex interactions in metacommunities, with implications for
	biodiversity and higher levels of selection.
	\newblock \emph{Ecology}, 73, 1984--2000.
	
\end{thebibliography}

